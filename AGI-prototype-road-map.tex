\input{../YKY-preamble.tex}
\setmainfont[BoldFont=Alibaba_Sans_Regular.otf,ItalicFont=Alibaba_Sans_Light_Italic.otf]{Alibaba_Sans_Light.otf}

% \usepackage[backend=biber]{biblatex}
% \bibliography{../AGI-book}

\usepackage[active,tightpage]{preview}		% for continuous page(s)
\renewcommand{\PreviewBorder}{0.5cm}
\renewcommand{\thempfootnote}{\arabic{mpfootnote}}

\usepackage[absolute,overlay]{textpos}		% for page number on upper left corner

\usepackage{color}
\usepackage{mathtools}
\usepackage[hyperfootnotes=false]{hyperref}

% \usepackage[backend=biber,style=numeric]{biblatex}
% \bibliography{../AGI-book}
% \renewcommand*{\bibfont}{\footnotesize}

\usetikzlibrary{shapes}
\usepackage[export]{adjustbox}				% ??
\usepackage{verbatim} % for comments
% \usepackage{newtxtext,newtxmath}	% Times New Roman font

% \titleformat{\subsection}[hang]{\bfseries\large\color{blue}}{}{0pt}{} 
% \numberwithin{equation}{subsection}

\newcommand{\underdash}[1]{%
	\tikz[baseline=(toUnderline.base)]{
		\node[inner sep=1pt,outer sep=10pt] (toUnderline) {#1};
		\draw[dashed] ([yshift=-0pt]toUnderline.south west) -- ([yshift=-0pt]toUnderline.south east);
	}%
}%

\newcommand\reduline{\bgroup\markoverwith{\textcolor{red}{\rule[-0.5ex]{2pt}{0.4pt}}}\ULon}

%\DeclareSymbolFont{symbolsC}{U}{txsyc}{m}{n}
%\DeclareMathSymbol{\strictif}{\mathrel}{symbolsC}{74}
\DeclareSymbolFont{AMSb}{U}{msb}{m}{n}
\DeclareSymbolFontAlphabet{\mathbb}{AMSb}
% \setmathfont{Latin Modern Math}
\DeclareMathOperator*{\argmin}{arg\,min}

% \usepackage[most]{tcolorbox}
\tcbset{on line, 
	boxsep=4pt, left=0pt,right=0pt,top=0pt,bottom=0pt,
	colframe=red,colback=pink,
	highlight math style={enhanced}
}
\newcommand{\atom}{\vcenter{\hbox{\tcbox{....}}}}

\let\oldtextbf\textbf
\renewcommand{\textbf}[1]{\textcolor{blue}{\oldtextbf{#1}}}

\newcommand{\logic}[1]{{\color{violet}{\textit{#1}}}}
\newcommand{\underconst}{\includegraphics[scale=0.5]{../2020/UnderConst.png}}
\newcommand{\KBsymbol}{\vcenter{\hbox{\includegraphics[scale=1]{../KB-symbol.png}}}}
\newcommand{\token}{\vcenter{\hbox{\includegraphics[scale=1]{token.png}}}}
\newcommand{\proposition}{\vcenter{\hbox{\includegraphics[scale=0.8]{proposition.png}}}}

\newcommand{\circled}[1]{{\textcircled{\sffamily \scriptsize{#1}}}}

\begin{document}

\begin{preview}

\cc{
\title{\vspace{-1.5cm} \bfseries\color{blue}{\LARGE AGI Prototype Road Map}}
}{
\title{\vspace{-1.5cm} \bfseries\color{blue}{\LARGE AGI Prototype Road Map}}
}

% \author{YKY} % Your name
\date{\vspace{-2cm}} % Date, can be changed to a custom date

\maketitle

\setcounter{section}{-1}

% (1) Circled page number on upper left corner
% \begin{textblock*}{5cm}(2.1cm,2.3cm) % {block width} (coords) 
% {\color{red}{\large \textcircled{\small 1}}}
% \end{textblock*}

\begin{minipage}{\textwidth}
\setlength{\parskip}{0.4\baselineskip}

可以分拆成以下较细问题:
\begin{itemize}
	\item [\circled{1}] \textbf{选择机制}: Transformer 输出 $N$ 个 tokens,每个 token 有它的 Q-value,我们选择最高 Q-值 的 tokens 放进 工作记忆。 这似乎没有什么困难。

	\item [\circled{1}{\small B}] \textbf{encode Q-值}: token 通常是一个 512-维向量,我认为应该增加维数(因为命题空间明显应该比 词汇空间更大)。 透过矩阵乘法,一个 token 可以变成词汇之上的概率分布。 

	\item [\circled{1}{\small C}] \textbf{工作记忆 cache}: 因为 Transformer 每次输出一大堆 candidates,工作记忆 遗忘的速度不够快,似乎需要 caching.

	\item [\circled{2}] \textbf{「找关系」函数}: 设计一个 $M$-in $N$-out $c$ 函数 $c$ = ``chooser'',$M$ 和 $N$ 是 tokens 个数,$M \gg N$。 它的作用是 从 $M$ 个 tokens 里找出最有可能是 某个逻辑 rule 的前提的 $N$ 个 tokens,作为 Transformer 的输入。 注意: 每个 rule 的前提都是不同的,所以 $c$ 的输出 可以有很多可能。 这个问题其实在 LLM 里已经有研究,因为我们不需要 Transformer 有太长的遥距相关。 
	
	\item [\circled{3}] \textbf{遗忘机制}: 在工作记忆里,如何选出那些要被遗忘的 tokens?
	
	\item [\circled{4}] \textbf{输入/输出 编码方法}: 工作记忆里面,所有 tokens = 命题 都是没有次序的。 那么需要一个位置编码的方法,这个跟以前 Transformer 的要求是一样的。
	
	\item [\circled{5}]  \textbf{RL 的训练方法}: 在 强化学习下,输入新的句子后,我们的 RL 系统会在 $n$ 步之后输出 词语,组成句子。我们要用 GPT 给予 RL 系统奖励。
	
\end{itemize}

\end{minipage}
\end{preview}
\end{document}
