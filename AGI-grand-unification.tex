\input{../YKY-preamble.tex}
\setmainfont[BoldFont=Alibaba_Sans_Regular.otf,ItalicFont=Alibaba_Sans_Light_Italic.otf]{Alibaba_Sans_Light.otf}
	
\usepackage[active,tightpage]{preview}		% for continuous page(s)
\renewcommand{\PreviewBorder}{0.5cm}
\renewcommand{\thempfootnote}{\arabic{mpfootnote}}

\usepackage[absolute,overlay]{textpos}		% for page number on upper left corner

\usepackage{color}
\usepackage{mathtools}
\usepackage[hyperfootnotes=false]{hyperref}

% \usepackage[backend=biber,style=numeric]{biblatex}
% \bibliography{../AGI-book}
% \renewcommand*{\bibfont}{\footnotesize}

\usetikzlibrary{shapes}
\usepackage[export]{adjustbox}				% ??
\usepackage{verbatim} % for comments
% \usepackage{newtxtext,newtxmath}	% Times New Roman font

% \titleformat{\subsection}[hang]{\bfseries\large\color{blue}}{}{0pt}{} 
% \numberwithin{equation}{subsection}

\newcommand{\underdash}[1]{%
	\tikz[baseline=(toUnderline.base)]{
		\node[inner sep=1pt,outer sep=10pt] (toUnderline) {#1};
		\draw[dashed] ([yshift=-0pt]toUnderline.south west) -- ([yshift=-0pt]toUnderline.south east);
	}%
}%


%\DeclareSymbolFont{symbolsC}{U}{txsyc}{m}{n}
%\DeclareMathSymbol{\strictif}{\mathrel}{symbolsC}{74}
\DeclareSymbolFont{AMSb}{U}{msb}{m}{n}
\DeclareSymbolFontAlphabet{\mathbb}{AMSb}
% \setmathfont{Latin Modern Math}
\DeclareMathOperator*{\argmin}{arg\,min}

% \usepackage[most]{tcolorbox}
\tcbset{on line, 
	boxsep=4pt, left=0pt,right=0pt,top=0pt,bottom=0pt,
	colframe=red,colback=pink,
	highlight math style={enhanced}
}
\newcommand{\atom}{\vcenter{\hbox{\tcbox{....}}}}

\let\oldtextbf\textbf
\renewcommand{\textbf}[1]{\textcolor{blue}{\oldtextbf{#1}}}

\newcommand{\logic}[1]{{\color{violet}{\textit{#1}}}}
\newcommand{\underconst}{\includegraphics[scale=0.5]{../2020/UnderConst.png}}
\newcommand{\KBsymbol}{\vcenter{\hbox{\includegraphics[scale=1]{../KB-symbol.png}}}}
\newcommand{\token}{\vcenter{\hbox{\includegraphics[scale=1]{token.png}}}}
\newcommand{\proposition}{\vcenter{\hbox{\includegraphics[scale=0.8]{proposition.png}}}}

\begin{document}

\begin{preview}

\cc{
\title{\vspace{-1.5cm} \bfseries\color{blue}{\LARGE AGI 大统一理论}}
}{
\title{\vspace{-1.5cm} \bfseries\color{blue}{\LARGE AGI Grand Unification}}
}

% \author{YKY} % Your name
\date{\vspace{-2cm}} % Date, can be changed to a custom date

\maketitle

\setcounter{section}{-1}

% (1) Circled page number on upper left corner
\begin{textblock*}{5cm}(2.1cm,2.3cm) % {block width} (coords) 
{\color{red}{\large \textcircled{\small 1}}}
\end{textblock*}

\begin{minipage}{\textwidth}
\setlength{\parskip}{0.4\baselineskip}

\section{综述}

\newcommand\reduline{\bgroup\markoverwith{\textcolor{red}{\rule[-0.5ex]{2pt}{0.4pt}}}\ULon}

\begin{itemize}

	\item 大统一理论是在 \textbf{强化学习} 的框架下进行的,这是以 Richard Sutton 为代表人物 提出的理论框架。 

	\item 在 强化学习里 最辣手的一个问题,就是如何 储存和计算 所有 \textbf{状态} 之上的 \textbf{概率分布}。 对 AGI 来说,状态 = 思维空间。 我们需要的是 所有可能的思维之上的概率分布,而这是 AGI 的一个硬性需求,无法避免。 由于思维空间是高维的向量空间,它上面的概率分布是一个庞大的 mathematical object,很难在计算机上表示。 如果用 神经网络 表示,则问题是如何对这个概率分布进行 \textbf{采样} (sampling), 在神经网络里,这是很困难的。

	\item \textbf{Hopfield 网络}的权重 定义了一个 能量地势 (energy landscape),它可以看成是一个 implicit 的 \textbf{概率分布}。 透过 Hopfield 网络的 learning,可以改变这个概率分布。 \reduline{但这需要修改 Hopfield 网络的算法,将 能量 诠释成 概率},而这正是 \textbf{Boltzmann machine}.

	\item 根据 ``Hopfield Network is All You Need'' 论文,现代 Hopfield 网络的 state update rule 跟 \textbf{Transformer} 重合\footnote{注意这是 state update rule 而不是 learning update rule.  前者 更改 Hopfield 网络的 激活 状态; 后者 更改 Hopfield 网络的权重/记忆。}。 换句话说,每执行一次 Transformer,就会趋向 Hopfield 的能量最低点。 

	\item Transformer 的 softmax 可以看成是 \textbf{大脑}中某种 ``winner-takes-all'' 机制。 从这个角度,可以诠释大脑思考的机制。
	
	\item 我最新的论文 提出,Transformer 具有 \textbf{逻辑结构},可以在逻辑基础上建立 AGI.

\end{itemize}

\section{Modern Hopfield networks}

在经典 Hopfield 网络里,当 A 和 B 两个 patterns 太靠近的时候,它们会互相干扰,导致可以储存的 patterns 数量不大。  现代 Hopfield 网络 改变 Hamiltonian 能量函数,令干扰减弱,可以储存数量更多的 patterns:
\begin{equation}
\vcenter{\hbox{\includegraphics[scale=0.7]{modern-Hopfield-network-energy-landscape.png}}}
\end{equation}


\section{大脑}



\end{minipage}
\end{preview}

\begin{comment}
\begin{preview}
\begin{minipage}{\textwidth}
\setlength{\parskip}{0.4\baselineskip}

\begin{textblock*}{20cm}(2.1cm,2cm) % {block width} (coords) 
	{\color{red}{\large \textcircled{\small 2}}}
	\hspace{8cm}
	\color{blue}{\footnotesize \cc{逻辑 Transformer}{Logic Transformer}}
\end{textblock*}
\vspace*{0.3cm} 

\end{minipage}
\end{preview}
\end{comment}

\end{document}
